\documentclass[a4paper,12pt]{book}
\input{~/Box/LaTeX_Header/header.tex}

\title{O-minimal geometry and Hodge Theory}
\author{Shuddhodan Kadattur Vasudevan}
\date{\today}

\begin{document}

\maketitle

% Your content here

\chapter{Introduction}
\label{chap:intro}

Consider a smooth projective variety $X/\bbC$ and let $X^{\mathrm{an}}:=X(\bbC)$ be the associated compact complex manifold. Let $H^{k}_{B}(X,\bbZ)$ be the $k^{\mathrm{th}}$-Betti cohomology (with $\bbZ$-coefficients) of $X^{\mathrm{an}}$. Then classical Hodge theory  
can be summarized by saying that $H^{k}_{B}(X,\bbZ)$ has a \textit{polarized Hodge structure of weight} $k$. Here is what it means:

\begin{enumerate}[(a)]
\item There exists a \textit{Hodge decomposition}
\[
H^{k}_{B}(X,\bbZ) \otimes_{\bbZ} \bbC \simeq \bigoplus_{p+q=k}H^{p,q},
\]
\noindent here $H^{p,q}$ are complex vector space satisfying $(1 \otimes \sigma)H^{p,q}=H^{q,p}$\footnote{Henceforth we shall denote $(1 \otimes \sigma)$ by the bar operation.}, where $\sigma \in \text{Aut}(\bbC)$ is the complex conjugation.  

\item There exists a $(-1)^{k}$-symmetric bilinear form 
\[
q : H^{k}_{B}(X,\bbZ) \otimes_{\bbZ}H^{k}_{B}(X,\bbZ) \to \bbZ,
\]
\noindent such that $q(\alpha,\beta)=$\footnote{We continue to denote the base extension of $q$ to complex numbers by $q$.} for $\alpha \in H^{p,q}$ and $\beta \in H^{p',q'}$ unless $p=q'$ and $q=p'$. Moreover for any $\alpha \in H^{p,q}\backslash \{0\}$,

\[
(-1)^{q-p}q(\alpha,\bar{\alpha}) >0.
\] 
\end{enumerate}

Here is a nice corollary.

\begin{cor}\label{cor:odd-betti-even}
Let $X/\bbC$ be a smooth projective variety. Then the odd Betti numbers are even dimensional.
\end{cor}

\begin{proof}
Follows from the Hodge decomposition above. 
\end{proof}

\begin{example}\label{ex:Hopf-surface}
Every one dimensional compact complex manifold is algebraic. However in dimension $2$ this fails. 

For example let $\tilde{X}=\bbC^{2}\backslash \{0\}$. Let $\lambda$ be a non-zero complex number which is \textit{not} a root of unity. Let $\Gamma=\bbZ$ act on $\tilde{X}$ via scaling by $\lambda$. Then the action of $\Gamma$ on $\tilde{X}$ is free, properly discontinuous and via holomorphic automorphisms. Thus $X := \Gamma\backslash\tilde{X}$ is also a complex manifold of dimension $2$. By choosing an appropriate closed shell in $\bbC^{2} \backslash \{0\}$, we may realize $X$ as the continuous image of a compact set. Thus $X$ is a compact complex manifold. Evidently $\tilde{X}$ is the universal covering space of $X$ and $\pi_{1}(X) \simeq \bbZ$. In particular $H^{1}_{B}(X,\bbZ)=\bbZ$. Thus $X$ cannot have an algebraic structure!

\end{example}


\subsection{Hodge theory over a point}
\label{sec:hodge-theory-point}


Before we proceed here is a quick summary of how to equip $H^{k}_{B}(X,\bbZ)$ with a polarized Hodge structure of weight $k$.

\subsubsection{Equipping $H^k_B(X,\bbZ)$ with a Hodge structure}
\label{sec:equipping-HS}

We equip $H^{k}_{B}(X,\bbZ)$ with a Hodge structure of weight $k$ using the de Rham isomorphism and Hodge theory. The former gives us an isomorphism 
\[
H^{k}_{B}(X,\bbZ) \otimes_{\bbZ} \bbC \simeq H^{k}_{dR}(X^{\mathrm{an}},\bbC).
\]
Here $H^{k}_{dR}(X^{\mathrm{an}},\bbC)$ is the de Rham cohomology defined to be the hyercohomology of the de Rham complex
\[
0 \to \sA^{0}_{X} \to \sA^{1}_{X} \cdots \to \sA^{2n}_{X} \to 0,
\]
\noindent where $\sA^{i}_{X}$ is the sheaf of (complex valued $\sC^{\infty}$) $i$-forms on $X^{\mathrm{an}}$ (which we assume has dimension $n$). Hodge theory allows us to further decompose $H^{k}_{dR}(X^{\mathrm{an}},\bbC)$ by \textit{type}, the upshot of which is an isomorphism
\[
H^{k}_{dR}(X^{\mathrm{an}},\bbC) \simeq \bigoplus_{p+q=k}H^{q}(X^{\mathrm{an}},\Omega^{p}_{X^{\mathrm{an}}}).\footnote{Here $\Omega^{*}_{X^{\mathrm{an}}}$ is the sheaf of K\"ahler differentials on the comapct complex manifold $X^{\mathrm{an}}$.}
\] 
It follows from Hodge theory and the de Rham isomorphism that if we set $H^{p,q}:=H^{q}(X^{\mathrm{an}},\Omega^{p}_{X^{\mathrm{an}}})$, then $\overline{H^{p,q}}=H^{q,p}$. In this course we will take this for granted\footnote{Though we will not need it, I strongly encourage you to read a proof of this decomposition. A possible reference is \cite[Chapter 6]{voisinHodgeTheory}.}.  

\subsubsection{Cycle class map}
\label{sec:cycle-class-map}

Before we can discuss about polarizations, let me briefly recall the cycle class map. Let $Z^{i}(X)$ denote the free abelian group generated by codimension $i$ sub varieties of $X$. The cycle class map is a morphism of abelian groups 
\[
\mathrm{cl}_{X}: Z^{i}(X) \to H^{2i}_{B}(X,\bbZ).
\]
Intuitively this is easy to define, start with any codimension $i$ sub variety $Z$. Hence $Z \subseteq X$ is a subvariety of dimension $n-i$ and thus defines a homology class in $H_{2n-2i}(X^{\mathrm{an}},\bbZ)$ which by Poincare duality gives us a class in $H^{2i}_{B}(X,\bbZ)$. It is not hard to see the following.

\begin{lem}\label{lem:cycle-class-decomposition}
For any subvariety $Z \subseteq X$ of codimension $i$
\[
\mathrm{cl}_{X}(Z) \in H^{i,i}\bigcap\mathrm{Im}(H^{2i}_{B}(X,\bbZ) \to H^{2i}_{B}(X,\bbZ)\otimes_{\bbZ}\bbC).
\]
\end{lem}

\begin{proof}
For a proof see \cite[Proposition 11.20]{voisinHodgeTheory}.
\end{proof}

The Hodge conjecture predicts that the converse must be true \textit{rationally}. More precisely.

\begin{conj}[Hodge Conjecture]
\label{conj:Hodge-conjecture}
The inclusion $\mathrm{Im}(\mathrm{cl}_{X})_{\bbQ} \subseteq H^{i,i}\cap H^{2i}_{B}(X,\bbQ)$ is an isomorphism.   
\end{conj}

\begin{rem}\label{rem:Hodge-conjecture}
Here are a few remarks about the Hodge conjecture.
\begin{enumerate}[(i)]
\item Unlike Lemma \ref{lem:cycle-class-decomposition}, we need not write $\mathrm{Im}$ since $H^{2i}_{B}(X,\bbQ)$ is a subspace of $H^{2i}_{B}(X,\bbZ) \otimes \bbC$.

\item The Hodge conjecture is false integrally. The earliest known counterexamples were due to Atiyah-Hirzebruch. A reference (with some history) is \cite{soule2005torsion}.

\end{enumerate}
\end{rem}

Let us end this section with a positive evidence for Hodge conjecture.

\begin{prop}[Lefschetz (1,1) theorem]
\label{prop:hodge-conjecture-divisors}
The Hodge conjecture is true for divisors (even integrally).
\end{prop}

\begin{proof}
Consider the exponential sequence
\[
\begin{tikzcd}
0 \ar[r] & \bbZ_{X} \ar[r,"{2\pi i}"] & \sO_{X} \ar[r,"\exp"] & \sO^{*}_{X} \ar[r] & 0.
\end{tikzcd}
\]

Here $\bbZ_{X}$ is the constant sheaf with values in $\bbZ$. The sheaves $\sO_{X}$ and $\sO_{X}^{*}$ are the sheaves of holomorphic functions with values in $\bbC$ and $\bbC^{*}$ respectively. Taking the long exact sequence on cohomology, the maps of interest are the boundary map 
\[
c_{1}: \mathrm{Pic}(X^{\mathrm{an}})=H^{1}(X,\sO_{X}^{*}) \to H^{2}_{B}(X,\bbZ)
\]

\noindent and

\[
\tau_{*}: H^{2}_{B}(X,\bbZ) \to H^{2}(X,\sO_{X}).
\]

Using a standard argument, we can show that the cycle class map factors through this boundary map $c_{1}$ and that $Z^{1}(X) \twoheadrightarrow \text{Pic}(X^{\mathrm{an}})$. Hence it suffices to show that $\text{ker}(\tau_{*})$ is precisely $H^{1,1}\bigcap\mathrm{Im}(H^{2}_{B}(X,\bbZ) \to H^{2}_{B}(X,\bbZ)\otimes_{\bbZ}\bbC)$. This follows from the fact that $\tau_{*}$ can be identified with the composition 
\[
H^{2}_{B}(X,\bbZ) \to H^{2}_{B}(X,\bbC) \twoheadrightarrow H^{0,2}=H^{2}(X,\sO_{X}).
\]

For details we refer the reader to \cite[Pg 163]{GriffithsHarris}
\end{proof}


\subsubsection{Polarization on $H^k_B(X^\mathrm{an},\bbZ)$}
\label{sec:polarization-equip}

The polarization on $H^{k}_{B}(X,\bbZ)$ comes from the class of a hyperplane section corresponding to any embedding of $X \hookrightarrow \bbP^{n}_{\bbC}$. Indeed let $\omega \in H_{B}^{2}(X^{\mathrm{an}},\bbZ)$ the class corresponding to an hyperplane section. Then the bilinear form 
\[
q : H^{k}_{B}(X,\bbZ) \otimes H^{k}_{B}(X,\bbZ) \to \bbZ
\] 

\noindent is defined to be 
\[
q(\alpha,\beta):=\text{Tr}(\alpha \cup \beta \cup \omega^{k}) 
\]
\noindent induces a polarization on $H^{k}_{B}(X,\bbZ)$. For a proof we refer the reader to \cite[Theorem 6.32]{voisinHodgeTheory}

\subsubsection{What about non-smooth/non-proper varieties?}
\label{sec:non-smooth-nonproper}


As we shall see later in the course, just starting with these structures on the cohomology of a smooth projective variety we can extrapolate and obtain more complex structures on the cohomology of \textit{any} complex algebraic variety. Moreover any morphism between such varieties induces a morphism on their cohomology respecting these additional structures. If you play this game well there is much you can say by linearizing the problem of understanding varieties and morphisms between them.

\subsection{Hodge Theory over a base}
\label{sec:Hodge-Theory-base}

A basic lesson one learns over time when using Algebraic geometry is that its best to set things up over a base. Often we need to deform or lift varieties or move in families even though we might be interested in proving results about a particular variety. Hodge theory in its modern avatar handles this with ease. To begin with lets recall Ehresmann fibration theorem (see \cite[Theorem 9.3]{voisinHodgeTheory}). 

\begin{thm}\label{thm:ehresmann}
Let $f : M \to N$ be a smooth and proper map between smooth manifolds, then $f$ is locally on $N$ a fiber bundle.
\end{thm}

Here is a nice corollary.

\begin{cor}\label{cor:betti-cohomology-constant}
Let $f : M \to N$ be a proper and smooth morphism of smooth manifolds. Then $R^{i}f_{*}\bbZ_{M}$ is a local system on $N$ i.e. is locally a constant sheaf. Moreover if $N$ is connected then all the fibers have the same Betti rank. 
\end{cor}


The upshot is that if you have a smooth and proper family then topologically not much changes when you move from one fiber to another. However this is false holomorphically/algebraically. Here is a standard example.

\begin{example}\label{ex:Legendre-family}
Consider the Legendre family of elliptic curve
\[
\pi : \sE \to \Delta\backslash\{0,1\}.
\]

Here $\sE \subseteq \bbP^{2} \times \Delta^{*}$ is given by the vanishing of 
\[
Y^{2}Z=X(X-Z)(X-tZ).
\]
Then $\pi$ is a smooth and proper family. But the $j$-invariant of these elliptic curves is a function of $t$ and is in particular not constant. Thus the Hodge structure (the only relevant one here is $H^{1})_{B}(\sE_{t},\bbZ)$) varies along the base though the underlying $\bbZ$-module remains the same. This is a typical example of a \textbf{variation of Hodge structures}.
\end{example}

One way to think about Example \ref{ex:Legendre-family} is to choose a point say $t_{0}$ and fix the underlying $\bbZ$-module for the variation of Hodge structure say $V=H^{1}(\sE_{t_{0}},\bbZ)$. As $t$ varies the Hodge structure on $V$ varies. But how do we capture this? 

One way to do this would be to look at the space of all Hodge decompositions on $V$ and then assign to $t$ the corresponding Hodge decomposition. Of course we would want this map to be at least holomorphic. But this fails for the simple reason that $H^{1,0}$ and $H^{0,1}$ are complex conjugates, so any attempt at making one of the holomorphic would force the other to be anti holomorphic and hence the resulting map would at best be real analytic. The remedy? Well we can ignore the anti holomorphic part since it is essentially determined by the holomorphic part. 

\subsubsection{Variation of Hodge structures}
\label{sec:variation-of-hodge-structures}


More generally given a family $\pi : X \to S$ of smooth projective varieties, we obtain for any $k$ a variation of Hodge structures on the lattice $V=H^{k}_{B}(X_{s},\bbZ)$. One possible way to capture the change (or variation) in the Hodge structure as one moves along $S$ is by looking at the \textit{period map}\footnote{The name is justified by the Legendre family example, where the variation is literally captured by the period.}.
\[
\Phi : S \to D\footnote{This is not correct as written, but is a very good approximation. We shall make this precise eventually.},
\]
\noindent here $D$ is the space of all possible Hodge structures on $V$, also called a \textit{period domain}. Thus $\Phi$ captures the Hodge theoretic complexity of $f$ (or its fibers). 

To see why this is useful, suppose the Hodge conjecture were true. Then the fibers of $f$ which have extra algebraic cycles of codimension $k$, are those for which $H^{k,k}$ is particularly large. This in turn is captured by looking at the period map for $2k$ and looking at special points in its period domain i.e. those for which the corresponding Hodge structure has a bigger $H^{k,k}$.

\subsubsection{What can we say about the period map?}
\label{sec:what-say-period-map}

Even though we started with algebraic objects like $\pi, X$ and $S$, the period domain itself is rarely algebraic. The best one can do is put an analytic structure on $D$ and consequentially $\phi$ is a morphism of analytic spaces. This is where o-minimal geometry enters the picture. 


One of the aims of this course is to show that period domains are semi-algebraic or equivalently definable in $\bbR_{\mathrm{alg}}$, (the simplest o-minimal structure) in a functorial way which we shall make precise later. However the period map $\Phi$ itself is more complicated. For example the period map for the Legendre family involves hypergeometric functions. So one cannot expect any algebraicity for $\Phi$. In fact one of the main results of \cite{BakkerKlinglerTsimerman2020} is the following:

\begin{thm}[Bakker-Klingler-Tsimerman]\label{thm:r-an-exp-definability}
The period maps $\Phi$ are definable in $\bbR_{\mathrm{an},\exp}$.
\end{thm}
 
This result is amplified in strength by the following o-minimal GAGA by Peterzil-Starchenko. 

\begin{thm}[Peterzil-Starchenko]\label{thm:o-minimal-GAGA}
Let $S$ be a quasi-projective complex variety and let $Z \subset S$ be a closed analytic subset. If there exists an o-minimal structure
expanding $\bbR_{\mathrm{an}}$ in which $Z$ is definable, then $Z$ is algebraic.
\end{thm}

Combining Theorems \ref{thm:r-an-exp-definability}, \ref{thm:o-minimal-GAGA} and a key algebraization result for definable images allows the authors to prove Griffiths conjecture on algebraicity of the images of period maps. 



\chapter{Hodge Structures}
\label{chap:hodge-structures}

In this chapter we shall discuss the necessary background from Hodge theory. As we mentioned in class, we are going to take Hodge decomposition on compact K\"ahler manifolds (or smooth projective varieties) for granted. That this is not a buzzkill is to me a feature of the theory than a bug.   

\section{Pure Hodge structures}
\label{sec:pure-HS}


\bibliographystyle{plain}
\bibliography{references.bib}

\end{document}
